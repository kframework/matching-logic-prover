\documentclass{article}

\title{Fixpoints in Matching Logic}
\author{Formal Systems Laboratory}

\begin{document}
\maketitle

Fixpoints are solutions to equations of the form
$$
X = F(X)
$$
where $X$ is the variable of the equation and $F(X)$ is an expression about $X$.
Depending on what $X$ is, fixpoints are categorized as
\begin{itemize}
	\item element fixpoints, if $X$ ranges over elements;
	\item relation fixpoints, if $X$ ranges over relations;
	\item function fixpoints, if $X$ ranges over functions;
\end{itemize}
In addition, if there is a partial order $\leq$ on the domain where $X$ ranges, 
we can define least fixpoints w.r.t. the order $\leq$ as follows
\begin{center}
	$X = F(X)$ and for any $Y$ such that $Y = F(Y)$, we have $X \leq Y$.
\end{center}

Element fixpoints are definable in first order logic.
Given an equation
$$
x = f(x)
$$
where the right-hand side is a term in which the variable $x$ occurs free.
The following formula
$$
\exists x . x = f(x)
$$
is true in precisely those models where the interpretation of $f$ has an
fixpoint.
The following formula
$$
\exists x . ( x = f(x) \wedge \forall y . y = f(y) \to x \le y )
$$
is true in precisely those models where the interpretation of $f$ has a least 
fixpoint.
Let $e$ be a constant symbol, the following formula
$$
 e = f(e) \wedge \forall y . y = f(y) \to e \le y 
$$
is true in precisely those models where the interpretation of $f$ has a least 
fixpoint, which is the interpretation of $e$.

Relation fixpoints are definable in first order logic.

\end{document}
 