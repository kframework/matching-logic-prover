\documentclass{article}

\theoremstyle{definition}
\newtheorem{definition}{Definition}[section]
\newtheorem{notation}[definition]{Notation}

\newcommand{\CoqTerm}{\textnormal{\textit{Term}}\xspace}
\newcommand{\CoqSort}{\textnormal{\textit{Sort}}\xspace}

\newcommand{\SProp}{\mathsf{SProp}}
\newcommand{\Prop}{\mathsf{Prop}}
\newcommand{\Set}{\mathsf{Set}}
\newcommand{\Type}{\mathsf{Type}}
\newcommand{\NN}{\mathbb{N}}
\newcommand{\V}{V}
\newcommand{\C}{C}
\newcommand{\cln}{{\,:\,}}
\newcommand{\scln}{\mathbin{;}}
\newcommand{\dcln}{\mathbin{::}}
\newcommand{\FV}{\mathrm{FV}}
\newcommand{\sbs}[3]{#1 [ #2 / #3 ]}

\newcommand{\cfa}[3]{\forall #1 \cln #2 , #3}
\newcommand{\clm}[3]{\lambda #1 \cln #2 . #3}
\newcommand{\cpp}[2]{(#1 \, #2)}
\newcommand{\clt}[4]{\textsf{let} \, #1 \coloneqq #2 \cln #3 \,\textsf{in}\, #4}

\newcommand{\lasm}[2]{#1 \cln #2}
\newcommand{\ldef}[3]{#1 \coloneqq #2 \cln #3}

\title{Proof Checking in Applicative Matching Logic}
\author{Xiaohong Chen}
\date{}

\begin{document}
\maketitle
\begin{abstract}
This paper discusses proof checking in applicative matching logic.
\end{abstract}



\section{Proof checking}

At a high level, \emph{proof objects} are strings that encode
a Hilbert-proof, including its underlying theory $\Gamma$,
the proof target $\varphi$, and all intermediate proof steps,
which are accompanied with detailed proof annotations that
specify which proof rules are applied and how they are instantiated
in every step.
We let $A$ to denote the underlying alphabet of proof objects.

\begin{itemize}
\item A proof object $\pi$ is called \emph{well-formed}, if it is a correct 
encoding of a proof;
\item A proof object $\pi$ is called \emph{valid}, if it is \emph{well-formed}
and it encodes a correct proof;
\item A proof object $\pi$ is called \emph{checked}, if it passes the proof 
checker.
\end{itemize}

\paragraph{Key property of proof checking.}
All checked proof objects are valid.


%\begin{itemize}
%\item A \emph{judgment}, written $\Gamma \vdash \varphi$,
%consists of a pattern set $\Gamma$ and a patter $\varphi$.
%\item A \emph{$\Gamma$-proof} is a sequence $\varphi_1,\dots,\varphi_n$
%such that every $\varphi_i$ for $1 \le i \le n$ is a logical axiom,
%or a pattern in $\Gamma$, or the result of applying a proof rule
%to some patterns from $\varphi_1,\dots,\varphi_{i-1}$.
%\item The judgment $\Gamma \vdash \varphi$ is \emph{provable},
%if there exists a $\Gamma$-proof $\varphi_1,\dots,\varphi_n$ where
%$\varphi_n \equiv \varphi$.
%\item A \emph{proof object} is a string (using alphabets in $A$) that encodes 
%a 
%$\Gamma$-proof.
%\item The string $\omega \in A^*$ is \emph{well-formed} 
%\end{itemize}

\end{document}