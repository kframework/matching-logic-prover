\documentclass{article}

\usepackage{amsmath,amssymb,amsthm}
\usepackage{fullpage}
\usepackage{hyperref}
\usepackage{longtable}
\usepackage{mathtools}
\usepackage{prftree}
\usepackage[dvipsnames]{xcolor}
\usepackage{xspace}

\theoremstyle{definition}
\newtheorem{definition}{Definition}[section]
\newtheorem{notation}[definition]{Notation}

\newcommand{\CoqTerm}{\textnormal{\textit{Term}}\xspace}
\newcommand{\CoqSort}{\textnormal{\textit{Sort}}\xspace}

\newcommand{\SProp}{\mathsf{SProp}}
\newcommand{\Prop}{\mathsf{Prop}}
\newcommand{\Set}{\mathsf{Set}}
\newcommand{\Type}{\mathsf{Type}}
\newcommand{\NN}{\mathbb{N}}
\newcommand{\V}{V}
\newcommand{\C}{C}
\newcommand{\cln}{{\,:\,}}
\newcommand{\scln}{\mathbin{;}}
\newcommand{\dcln}{\mathbin{::}}
\newcommand{\FV}{\mathrm{FV}}
\newcommand{\sbs}[3]{#1 [ #2 / #3 ]}

\newcommand{\cfa}[3]{\forall #1 \cln #2 , #3}
\newcommand{\clm}[3]{\lambda #1 \cln #2 . #3}
\newcommand{\cpp}[2]{(#1 \, #2)}
\newcommand{\clt}[4]{\textsf{let} \, #1 \coloneqq #2 \cln #3 \,\textsf{in}\, #4}

\newcommand{\lasm}[2]{#1 \cln #2}
\newcommand{\ldef}[3]{#1 \coloneqq #2 \cln #3}

\title{All Coq Rules in One Place}
\author{Xiaohong Chen \and Lucas Pe{\~{n}}a}

\begin{document}

\maketitle

\begin{abstract}
This document summarizes all the proof rules of the Coq proof assistant,
as listed in 
\url{https://coq.inria.fr/distrib/current/refman/language/cic.html}.
\end{abstract}

\section{Syntax}

Let us fix a countably infinite set $\V$ of \emph{variables}, denoted 
$x,y,\dots$. 
Let us fix a countably infinite set $\C$ of \emph{constants}, denoted
$c,d,\dots$.

\begin{definition}
We define the set \CoqTerm to be the smallest set that satisfies the following 
conditions:
\begin{enumerate}
\item $\SProp,\Prop,\Set \in \CoqTerm$; $\Type(i) \in \CoqTerm$ for every $i 
\in \NNp$. 
\item $\V \subseteq \CoqTerm$.
\item $\C \subseteq \CoqTerm$.
\item If $x \in \V$ and $T,U \in \CoqTerm$, 
      then $\cfa{x}{T}{U} \in \CoqTerm$.
\item If $x \in \V$ and $T,u \in \CoqTerm$,
      then $\clm{x}{T}{u} \in \CoqTerm$.
\item If $t,u \in \CoqTerm$,
      then $\cpp{t}{u} \in \CoqTerm$, called \emph{application}.
\item If $x \in \V$ and $t,T,u \in \CoqTerm$,
      then $\clt{x}{t}{T}{u} \in \CoqTerm$.
\end{enumerate}
where $\cfa{x}{T}{U}$ binds $x$ to $U$
and $\clm{x}{T}{u}$ binds $x$ to $u$.
We use $\FV(T) \subseteq \V$ to denote the set of free variables in $T \in 
\CoqTerm$. 
For $T,U \in \CoqTerm$ and $x \in \V$, 
we use $\sbs{T}{U}{x}$ to denote the result of substituting $U$ for $x$ in $T$,
where $\alpha$-renaming happens implicitly to prevent variable capture.
\end{definition}

\begin{definition}
We define the set 
$\CoqSort = \{\SProp,\Prop,\Set\} \cup \{\Type(i) \mid i \in \NN\}$.
Note that $\CoqSort \subseteq \CoqTerm$. 
Elements in $\CoqSort$ are called \emph{sorts} and denoted as $s$,
possibly with subscripts.
\end{definition}

\begin{definition}
A \emph{local assumption} is written $\lasm{x}{T}$, where $x \in \V$ and $T \in 
\CoqTerm$. 
A \emph{local definition} is written $\ldef{x}{u}{T}$,
where $x \in \V$ and $u,T \in \CoqTerm$.
In both cases, we call $x$ the \emph{declared variable}. 
A \emph{local context} is an ordered list of local assumptions and local 
definitions, such that the declared variables are all distinct. 
We use $\Gamma$, possibly with subscripts, to denote local contexts. 
\end{definition}

\begin{notation}\label{not:ldcl}
We use the notation $[\lasm{x}{T} \scln \ldef{y}{u}{U} \scln \lasm{z}{V}]$
to denote the local context that consists of
the local assumption $\lasm{x}{T}$,
the local definition $\ldef{y}{u}{U}$
and the local assumption $\lasm{z}{V}$,
with the implicit requirement that $x,y,z$ are all distinct.
The empty local context is written as $[]$. 
Let $\Gamma$ be a local context. 
We write $x \in \Gamma$ to mean that $x$ is declared in $\Gamma$.
We write $(\lasm{x}{T}) \in \Gamma$ to mean that the local assumption
$\lasm{x}{T}$ is in $\Gamma$, or that
the local definition $\ldef{x}{u}{T}$ is in $\Gamma$ for some $u \in \CoqTerm$.
We write $(\ldef{x}{u}{T}) \in \Gamma$ to mean that
the local definition $\ldef{x}{u}{T}$ is in $\Gamma$.
We write $\Gamma \dcln ( \lasm{x}{T} )$ to denote the local context
that enriches $\Gamma$ with $\lasm{x}{T}$, with the implicit requirement that 
$x \not\in \Gamma$.
Similarly, we write 
$\Gamma \dcln ( \ldef{x}{u}{T} )$ to denote the local context
that enriches $\Gamma$ with $\ldef{x}{u}{T}$, with the implicit requirement 
that $x \not\in \Gamma$.
We write $\Gamma_1 \scln \Gamma_2$ to mean the local context
obtained by concatenating $\Gamma_1$ and $\Gamma_2$, with the implicit 
requirement that all variables declared in $\Gamma_2$ are not declared in 
$\Gamma_1$. 
\end{notation}

\begin{definition}
A \emph{global assumption} is written $(\gasm{c}{T})$,
with the parentheses, where $c \in \C$ and $T \in 
\CoqTerm$. 
A \emph{global definition} is written $\gldef{c}{u}{T}$,
where $c \in \C$ and $u,T \in \CoqTerm$.
In both cases, we call $c$ the \emph{declared constant}. 
A \emph{global environment} is an ordered list of global assumptions and global
definitions, and also \emph{declarations of inductive objects}, which are 
defined later. 
We use $E$, possibly with subscripts, to denote global environments.
\end{definition}

\begin{notation}
We use the notation $\gasm{c_1}{T} \scln \gldef{c_2}{u}{U} \scln \gasm{c_3}{V}$
to denote the local context that consists of
the global assumption $\gasm{c_1}{T}$,
the global definition $\gldef{c_2}{u}{U}$
and the global assumption $\gasm{c_3}{V}$.
The empty global context is written as $[]$. % TODO: is this right?
Let $E$ be a local context. 
We write $c \in E$ to mean that $c$ is declared in $E$.
We write $(\gasm{c}{T}) \in E$ to mean that the global assumption
$\gasm{c}{T}$ is in $E$, or that
the global definition $\gldef{c}{u}{T}$ is in $E$ for some $u \in \CoqTerm$.
We write $(\gldef{c}{u}{T}) \in E$ to mean that
the global definition $\gldef{c}{u}{T}$ is in $E$.
We write $E \scln \gasm{c}{T}$ to denote the global context
that enriches $E$ with $\gasm{c}{T}$.
Similarly, we write 
$E \scln \gldef{c}{u}{T}$ to denote the global context
that enriches $E$ with $( \gldef{c}{u}{T} )$.
\end{notation}

\begin{notation}
We write $E[\Gamma] \proves \lasm{u}{T}$ to mean that $u$ is \emph{well-typed} 
with type $T$ in global environment $E$ and local environment $\Gamma$.
We write $\wf(E)[\Gamma]$ to mean that the global environment $E$ is 
\emph{well-formed}
and $\Gamma$ is a \emph{valid local context} in $E$.
\end{notation}

\begin{definition}
A term $u$ is \emph{well-typed} in a global environment $E$ if there is a
local context $\Gamma$ and type $T$ such that
$E[\Gamma] \proves \lasm{u}{T}$ is derivable with the rules below.
\end{definition}

\section{Coq Rules}

In this section we list all Coq rules. 
A \emph{rule} consists of a set of \emph{premises} and one \emph{conclusion},
separated by a horizontal bar.
For readability, we put \emph{side conditions} alongside the premises.
Side conditions are typed in $\sd{\text{green}}$, to distinguish
from the premises.


\subsection{Basic Typing Rules}

There are 18 basic typing rules, as shown below. 


\begin{center}
\bgroup
\def\arraystretch{4}
\begin{longtable}{llp{5cm}}
\textbf{Names} & \textbf{Rules} & \textbf{Comments} \\\hline
\prule{W-Empty}&
$\prftree{\cdot}{\wf([])[]}$ &
The empty global environment is well-formed, and the empty local context is a 
valid local context in the empty global environment.\\
\prule{W-Local-Assum}&
$\prftree{E[\Gamma]\vdash T \cln s}
{\sd{s \in S}}
{\sd{x \not\in \Gamma}}
{\wf(E)[\Gamma \dcln (x \cln T)]}$&
The side condition $x \not\in \Gamma$ needs not to be specified because 
it is implicit
in the notation $\Gamma \dcln (x \cln T)$; see Notation~\ref{not:ldcl}.\\
\prule{W-Local-Def}&
$\prftree{E[\Gamma]\vdash t \cln T}
{\sd{x \not\in \Gamma}}
{\wf(E)[\Gamma \dcln (\ldef{x}{t}{T})]}$&
The side condition $x \not\in \Gamma$ needs not to be specified because 
it is implicitly implied 
by the notation $\Gamma \dcln (\ldef{x}{t}{T})$; see Notation~\ref{not:ldcl}.\\
\prule{W-Global-Assum}&
$
\prftree
{E[]\vdash T \cln s}
{\sd{s \in S}}
{\sd{c \not\in E}}
{\wf(E ; c \cln T)[]}
$ & \\
\prule{W-Global-Def} &
$
\prftree
{E[]\vdash t \cln T}
{\sd{c \not\in E}}
{\wf(E ; \gldef{c}{t}{T})[]}
$ & \\
\prule{Ax-SProp} &
$
\prftree{\wf(E)[\Gamma]}
{E[\Gamma] \vdash \SProp \cln \Type(1)}
$ &\\
\prule{Ax-Prop} &
$
\prftree{\wf(E)[\Gamma]}
{E[\Gamma] \vdash \Prop \cln \Type(1)}
$ &\\
\prule{Ax-Set} &
$
\prftree{\wf(E)[\Gamma]}
{E[\Gamma] \vdash \Set \cln \Type(1)}
$ &\\
\prule{Ax-Type} &
$
\prftree{\wf(E)[\Gamma]}
{E[\Gamma] \vdash \Type(i) \cln \Type(i+1)}
$ & Here, $i \in \NNp$ is any positive natural number. \\
\prule{Var} &
\multicolumn{2}{l}{
$
\prftree{\wf(E)[\Gamma]}
{\sd{\text{$(\lasm{x}{T} \in \Gamma)$, or $(\ldef{x}{t}{T}) \in \Gamma$ for 
some $t \in \CoqTerm$}}}
{E[\Gamma] \vdash x \cln T}
$} \\
\prule{Const} &
\multicolumn{2}{l}{
$
\prftree{\wf(E)[\Gamma]}
{\sd{\text{$(\lasm{c}{T} \in E)$, or $(\ldef{c}{t}{T}) \in E$ for 
some $t \in \CoqTerm$}}}
{E[\Gamma] \vdash c \cln T}
$} \\
\prule{Prod-SProp} &
$
\prftree
{E[\Gamma] \vdash T \cln s}
{\sd{s \in S}}
{E[\Gamma \dcln (x \cln T)] \vdash U \cln \SProp}
{E[\Gamma]\vdash \cfa{x}{T}{U} \cln \SProp}
$ &\\
\prule{Prod-Prop} &
$
\prftree
{E[\Gamma] \vdash T \cln s}
{\sd{s \in S}}
{E[\Gamma \dcln (x \cln T)] \vdash U \cln \Prop}
{E[\Gamma]\vdash \cfa{x}{T}{U} \cln \Prop}
$ &\\
\prule{Prod-Set} &
\multicolumn{2}{l}{
$
\prftree
{E[\Gamma] \vdash T \cln s}
{\sd{s \in \{\SProp,\Prop,\Set\}}}
{E[\Gamma \dcln (x \cln T)] \vdash U \cln \Set}
{E[\Gamma]\vdash \cfa{x}{T}{U} \cln \Set}
$
} \\
\prule{Prod-Type} &
\multicolumn{2}{l}{
$
\prftree
{E[\Gamma] \vdash T \cln s}
{\sd{s \in \{\SProp,\Type(i)\}}}
{E[\Gamma \dcln (x \cln T)] \vdash U \cln \Type(i)}
{E[\Gamma]\vdash \cfa{x}{T}{U} \cln \Type(i)}
$
} \\
\prule{Lam} &
$
\prftree
{E[\Gamma] \vdash \cfa{x}{T}{U} \cln s}
{\sd{s \in S}}
{E[\Gamma \dcln (x \cln T)] \vdash t \cln U}
{E[\Gamma] \vdash \clm{x}{T}{t} \cln \cfa{x}{T}{U}}
$ &\\
\prule{App} &
$
\prftree
{E[\Gamma] \vdash t \cln \cfa{x}{U}{T}}
{E[\Gamma] \vdash u \cln U}
{E[\Gamma] \vdash (t \, u) \cln T[u/x]}
$ &\\
\prule{Let} &
$\prftree
{E[\Gamma] \vdash t \cln T}
{E[\Gamma \dcln (\ldef{x}{t}{T})] \vdash u \cln U}
{E[\Gamma] \vdash \clt{x}{t}{T}{u} \cln U[t/x]}
$
\end{longtable}
\egroup
\end{center}

\subsection{Conversion Rules}
In this section, we define what it means for two Coq programs to be
\emph{intentionally equal}, or \emph{convertible}.

\begin{center}
\bgroup
\def\arraystretch{4}
\begin{longtable}{llp{5cm}}
\textbf{Names} & \textbf{Rules} & \textbf{Comments} \\\hline
\prule{Beta}&
$\prftree{E[\Gamma] \vdash ((\clm{x}{T}{t}) u) \red_{\beta} \sbs{t}{x}{u}}$&
We say that $\sbs{t}{x}{u}$ is the \emph{$\beta$-contraction} of
$((\clm{x}{T}{t}) u)$, and that $((\clm{x}{T}{t}) u)$ is the
\emph{$\beta$-expansion} of $\sbs{t}{x}{u}$ \\
&&
$\iota$-reduction rules to be defined later \\
\prule{Delta-Local}&
$\prftree
{\wf(E)[\Gamma]}
{(\ldef{x}{t}{T}) \in \Gamma}
{E[\Gamma] \vdash x \red_{\Delta} t}$&
Reducing variable defined in local context \\
\prule{Delta-Local}&
$\prftree
{\wf(E)[\Gamma]}
{(\gldef{c}{t}{T}) \in E}
{E[\Gamma] \vdash c \red_{\delta} t}$&
Reducing constant defined in global context \\
\prule{Zeta}&
$\prftree
{\wf(E)[\Gamma]}
{E[\Gamma] \vdash \lasm{u}{U}}
{E[\Gamma \dcln (\ldef{x}{u}{U})] \vdash \lasm{t}{T}}
{E[\Gamma] \vdash \clt{x}{u}{U}{t} \red_{\zeta} \sbs{t}{x}{x}}$&
Remove local definitions occurring in terms \\
\end{longtable}
\egroup
\end{center}
In addition to the above convertibility rules, we also allow identifying a term
$t$ of type $\cfa{x}{t}{U}$ with its \emph{$\eta$-expansion} $\clm{x}{T}{(t x)}$
for $x$ fresh in $t$. Note \emph{$\eta$-reduction} is deliberately not defined
(TODO: show example?).

\begin{notation}
We write $E[\Gamma] \vdash t \red u$ for the contextual closure of the rules
defined above. That is, $t$ reduces to $u$ with global environment $E$ and local
context $\Gamma$ with one of the previous reductions $\beta, \Delta, \delta,
\iota,$ or $\zeta$.
\end{notation}

\begin{definition}
Two terms are called \emph{irrelevant} if they share a common type of a strict
proposition $\gasm{A}{\SProp}$. Irrelevant terms can be identified.
\end{definition}

\begin{definition}
Two terms $t_1, t_2$ are called \emph{$\beta\delta\iota\zeta\eta$-convertible}, or
\emph{convertible}, or \emph{equivalent} in global environment $E$ and local
context $\Gamma$ iff there exists $t_1, t_2$ such that
\[ E[\Gamma] \vdash t_1 \red \ldots \red u_1 \text{  and  } E[\Gamma] \vdash t_2 \red \ldots \red u_2  \]
and either $u_1$ and $u_2$ are identical up irrelevant subterms, or they are
convertible up to $\eta$-exxpansion. We denote this as
$E[\Gamma] \vdash t_1 =_{\beta\delta\iota\zeta\eta} t_2$
\end{definition}

\end{document}
